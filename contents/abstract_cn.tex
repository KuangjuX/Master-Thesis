\section*{摘要}

随着深度学习模型的不断发展,特别是大语言模型(LLM)的兴起,模型结构日益复杂,对硬件性能的要求也越来越高。现有的深度学习编译器在处理硬件感知算法(如 FlashAttention)时,往往难以充分发挥硬件的潜力。基于有向无环图(DAG)的中间表示无法有效描述嵌套循环和复杂的内存层级交互,而基于多面体模型的编译器在硬件特定优化方面存在局限性。

为了解决上述问题,本文提出了一种名为 AffineGraph 的新型编译器框架。该框架引入了基于多内存层级数据流图的硬件感知抽象,通过仿射变换精确建模内存层级转换和计算模式。在此基础上,本文提出了基于分块策略的内核融合技术,通过自动化的图变换和解析式代价模型,探索最优的分块和融合策略,并论证了该设计向 NVIDIA Hopper 等新架构的可扩展性。最后,本文设计了高效的内核映射策略,通过基于原子块和微任务的层级执行模型,将优化后的数据流图映射为高性能的 CUDA 代码。

实验结果表明,AffineGraph 在 GEMM、融合 GEMM(最高 3.02 倍加速)、FlashAttention(最高 3.72 倍加速)以及 Block Sparse Attention 等关键负载上,均取得了与手写优化代码相当甚至更好的性能。此外,在 NVIDIA H800(Hopper 架构)上的实验验证了 AffineGraph 的架构可移植性,在内存受限操作上达到了接近峰值的内存带宽利用率(约 89\%)。

\textbf{关键词:} 深度学习编译器,硬件感知,数据流图,内核融合,代码生成

